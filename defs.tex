%!TEX root = main.tex
%%%%%%%%%%%%%%%%%%%%%%%%%%%%%%%%%%%%%%%%%%%%%%%%%%%%%%%%%%%%%%%%%%%%%%%%%%%%%%%%%%%%%%%%
\documentclass[prodmode,permissions]{acmsmall}

\usepackage[ruled]{algorithm2e}
\renewcommand{\algorithmcfname}{ALGORITHM}
\usepackage[numbers,sort&compress]{natbib} % for citet
\SetArgSty{textrm}  % for algorithm2e
\SetAlFnt{\small}
\SetAlCapFnt{\small}
\SetAlCapNameFnt{\small}
\SetAlCapHSkip{0pt}
\IncMargin{-\parindent}

%\doi{XXXXXXX.XXXXXXX}% TeXSupport

\newcommand{\DBA}[1]{{\color{blue} {\bf[ DBA}: #1 {\bf]}}}
\newcommand{\NDP}[1]{{\color{red} {\bf[ NDP}: #1 {\bf]}}}



% Any additional packages needed should be included after jmlr2e.
% Note that jmlr2e.sty includes epsfig, amssymb, natbib and graphicx,
% and defines many common macros, such as 'proof' and 'example'.
%
% It also sets the bibliographystyle to plainnat; for more information on
% natbib citation styles, see the natbib documentation, a copy of which
% is archived at http://www.jmlr.org/format/natbib.pdf

\usepackage{cite}


\usepackage[l2tabu, orthodox]{nag}
\usepackage{mathtools,datetime}
\mathtoolsset{
showonlyrefs=true % or false in draft mode
}
\usepackage{comment}
\usepackage{amsmath, subfigure, dsfont, amsfonts, amssymb, graphicx, enumerate, xspace, cancel, verbatim}
\usepackage[colorlinks,pdfpagelabels,citecolor=blue,plainpages=false]{hyperref}

\newcommand*{\bfrac}[2]{\genfrac{}{}{0pt}{}{#1}{#2}}

\newcommand{\fix}{\marginpar{FIX}}
\newcommand{\new}{\marginpar{NEW}}

\DeclareMathOperator*{\argmax}{argmax}
\DeclareMathOperator*{\weight}{weight}
\DeclareMathOperator*{\argmin}{argmin}
\DeclareMathOperator*{\argsup}{argsup}
\DeclareMathOperator*{\arginf}{arginf}
\DeclareMathOperator*{\diag}{diag}
\DeclareMathOperator*{\expec}{\mathbb E}
\DeclareMathOperator*{\exphat}{\hat{\mathbb E}}
\newcommand{\grad}{\operatorname{\nabla}}

\newcommand{\eod}{{${}$\\}}
\newcommand{\X}{{\mathbf X}}
\newcommand{\bS}{{\mathbf S}}
\newcommand{\x}{{\mathbf x}}
\newcommand{\bq}{{\mathbf q}}
\newcommand{\bu}{{\mathbf u}}
\newcommand{\bC}{{\mathbf C}}
\newcommand{\Pt}{{\mathbf P}}
\newcommand{\bD}{{\mathbf D}}
\newcommand{\bE}{{\mathbf E}}
\newcommand{\bz}{{\mathbf z}}
\newcommand{\bd}{{\mathbf d}}
\newcommand{\bi}{{\mathbf i}}
\newcommand{\bbf}{{\mathbf f}}
\newcommand{\bL}{{\mathbf L}}
\newcommand{\bbb}{{\mathbf b}}
\newcommand{\bG}{{\mathbf G}}
\newcommand{\bh}{{\mathbf h}}
\newcommand{\bo}{{\mathbf o}}
\newcommand{\ba}{{\mathbf a}}
\newcommand{\bZ}{{\mathbf Z}}
\newcommand{\y}{{\mathbf y}}
\newcommand{\G}{{\mathsf{G}}}
\newcommand{\g}{{\mathsf g}}
\newcommand{\cS}{{\mathcal{S}}}
\newcommand{\cT}{{\mathcal{T}}}
\newcommand{\V}{{\mathcal{V}}}
\newcommand{\bv}{{\mathbf{v}}}
\newcommand{\cP}{{\mathcal{P}}}
\newcommand{\cG}{{\mathcal{G}}}
\newcommand{\cU}{{\mathcal{U}}}
\newcommand{\cW}{{\mathcal{W}}}
\newcommand{\bA}{{\mathbf{A}}}
\newcommand{\bI}{{\mathbf{I}}}
\newcommand{\bB}{{\mathbf{B}}}
\newcommand{\bP}{{\mathbf{P}}}
\newcommand{\be}{{\mathbf{e}}}
\newcommand{\bp}{{\mathbf{p}}}
\newcommand{\cY}{{\mathcal{Y}}}
\newcommand{\cO}{{\mathcal{O}}}
\newcommand{\cX}{{\mathcal{X}}}
\newcommand{\cI}{{\mathcal{I}}}
\newcommand{\ccH}{{\mathcal{H}}}
\newcommand{\cC}{{\mathcal{C}}}
\newcommand{\xn}{{\mathbf{a}}}
\newcommand{\indic}{{\mathds 1}}


\newtheorem{thm}{Theorem}
\newtheorem{meta}{Metatheorem}
\newtheorem{cor}[thm]{Corollary}
\newtheorem{prop}[thm]{Proposition}
\newtheorem{lem}[thm]{Lemma}
\newtheorem{defn}{Definition}
\newtheorem{algol}{Algorithm}
\newtheorem{assn}{Assumption}
\newtheorem{claim}[thm]{Claim}
\newtheorem{rem}{Remark}
\newtheorem{eg}{Example}
\newtheorem{con}{Conjecture}
% Different font in captions


% Definitions of handy macros can go here
\newcommand{\fracpartial}[2]{\frac{\partial #1}{\partial  #2}}

% Heading arguments are {volume}{year}{pages}{submitted}{published}{author-full-names}

